\chapter{Realisieren}

Verschiedene vorkonfigurierte Pakete helfen den Bericht, speziell die Realisierung, ansprechend zu formatieren und gestalten:

Aufruf von einer Aktion über ein Menü:
\menu{Extras > Settings > Rulers} \\
Drücken von Tastenkombinationen:
\keys{CTRL + R} \\
Verzeichnispfade:
\directory{C:/Windows/system32/hosts.txt} \\
Quellcode:

\begin{codebox}[]
  \begin{minted}{javascript}
// Sortieren eines Arrays mit BubbleSort
let bubbleSort = (inputArr) => {
    let len = inputArr.length;
    for (let i = 0; i < len; i++) {
        for (let j = 0; j < len; j++) {
            if (inputArr[j] > inputArr[j + 1]) {
                let tmp = inputArr[j];
                inputArr[j] = inputArr[j + 1];
                inputArr[j + 1] = tmp;
            }
        }
    }
    return inputArr;
};
  \end{minted}
\end{codebox}

\section{Entwicklungsumgebung}
\subsection{Versionierung}
Für die Versionierung wird Git verwendet. Dabei wird GitHub als Remote-Repository verwendet. Das Repository mit
dem Source-Code kann unter \url{https://github.com/aneshodza/gnosis} gefunden werden. \newline
Damit lokal gearbeitet werden kann wird das Projekt mit \mintinline{bash}{git clone <ssh-link>} lokal geklont.

\subsection{IDE}
Als IDE wird vim mit verschiedenen Plugins verwendet. Bestimmte Sachen wurden in der \mintinline{bash}{.vimrc} Datei
konfiguriert, damit die Arbeit möglichst effizient ist. \newline
Diese Konfigurationen sind unter \newline
\url{https://github.com/aneshodza/.dotfiles/blob/ad87ee9ecc5588a59d66e211797792099569ca95/.vimrc} zu finden.

\subsection{CI/CD}
Für die CI/CD Pipeline wird SemaphoreCI verwendet. Das ist passend, da auch die PA sehr eng mit SemaphoreCI verbunden
ist. \newline

\begin{minipage}{\textwidth}
  \section{Aufsetzen des Projektes}
  Zu Beginn wird Arbeitspaket 7, Aufestzen des Projektes, implementiert. Dazu sind folgende Schritte zu befolgen:
  \begin{enumerate}
    \item Erstellen des Remote-Repository
    \item Mit dem README beginnen
    \item Erstellen des Plugins
    \item Aufsetzen der CI/CD Pipeline \newline
  \end{enumerate}
\end{minipage}

\begin{minipage}{\textwidth}
  \subsection{Erstellen des Repository}
  Das Projekt wird auf GitHub mit Git Versionsverwaltet, weshalb als Erstes ein GitHub Remote-Repository (auch einfach
  Repository genannt) erstellt werden muss. Das wird nach GitHub Dokumentation gemacht \cite{github_create_repo}.
  \subsubsection{Namensgebung}
  Das Repository sowie Projekt werden \enquote{gnosis} genannt. Dieser Name kommt aus dem Altgriechischen und bedeutet
  \enquote{Wissen}. Der Name wurde gewählt, da das Projekt dem Nutzer Wissen über den Development-Stand des Projektes
  vermitteln soll. 
\end{minipage}

\begin{minipage}{\textwidth}
  \subsection{Erstellen des README}

\end{minipage}
