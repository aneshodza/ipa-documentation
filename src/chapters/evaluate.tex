\chapter{Auswerten}
In diesem Kapitel wird die PA kritisch ausgewertet. Dabei wird auf die Erfüllung der Anforderungen eingegangen
und die Probleme, welche während der PA aufgetreten sind, beschrieben.
\section{Erfüllungsgrad der Anforderungen}
Das erhaltene Produkt dieser PA hat zwei Teile: Die Dokumentation und das Programm.
\subsection{Dokumentation}
Die Dokumentation stammt von einem Template \cite{Buhler_ipa-template_2022} ab, welches in LaTeX geschrieben
wurde: Eine ganze \enquote{Programmiersprache}, welche sich sehr gut für Dokumentationen eignet. Dinge wie der
Glossar oder das Abbildungsverzeichnis werden von LaTeX automatisch generiert. \newline
Dazu wurde auch sehr viel Wert auf das Einhalten des Kriterienkatalogs gelegt, weshalb die Dokumentation auch
sehr vollständig sein.
\subsection{Programm}
Das Programm ist ein, komplett von Redmine abgekoppeltes, Plugin. Es wurde mit ERB und Ruby geschrieben, während
die (mit 100\% coverage) Tests mit MiniTest geschrieben wurden. Auch hier werden alle Kriterien erfüllt, da die
Anforderungen als \enquote{Leitfaden} genutzt wurden. Mit einer guten Planung für die Umsetzung ging auch die
Umsetzung sehr schnell. \newline
Was nicht gelungen ist, ist das aktive Polling zu vermeiden. Das heisst, dass das Plugin nach einer SemaphoreCI
Request zu GitHub geht und diesen Dienst nach einer Commit History fragt. Dies wird unter Kapitel
\ref{sec:why_active_polling} genauer beschrieben.
\section{Reflexion}
