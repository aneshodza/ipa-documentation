\chapter{Expertengespräche}
\section{Erstes Expertengespräch}
\label{sec:first-expert-meeting}
Das erste Expertengespräch fand am \textbf{24. Mai 2023 um 17:30} statt. Es war ein remote Meeting auf Zoom.
Die Teilnehmer waren:
\begin{itemize}
  \item \textbf{Anes Hodza} (Kandidat)
  \item \textbf{Touseef Arif} (HEX)
\end{itemize}

\subsection{Ablauf des Meetings}
Zu Beginn stellten sich beide Teilnehmer vor. Anschliessend bat der HEX den Kandidaten darum, die Aufgabenstellung
zusammenzufassen. Der Kandidat erklärte, dass er ein Redmine Plugin erstellen soll, welches \gls{Issue}s mit Pull Requests
sowie Deployments verknüpft. Der HEX bestätige dies. \newline
Danach ging der HEX mit dem Kandidaten die persönlichen Anforderungen durch, um zu verifizieren, dass beide auf der
gleichen Seite sind. Dabei wurde auch klar, dass der Kandidat die Anforderungen verstanden hat. \newline
Als Nächstes wurde geklärt, was der momentane Status bezüglich der VF ist. Diese ändert sich eventuell noch. \newline
Schlussendlich erklärte der HEX dem Kandidaten, welchen Inhalt das Glossar haben soll: Firmeninterne Begriffe, welche
für aussenstehende nicht verständlich sind. Der Kandidat kann sich darauf verlassen, dass der HEX Fachbegriffe
versteht.

\section{Zweites Expertengespräch}
\label{sec:second-expert-meeting}
Das zweite Expertengespräch fand am \textbf{1. Juni 2023 um 10:30} statt. Auch dieses Meeting war remote auf Zoom.
Folgende Teilnehmer waren dabei:
\begin{itemize}
  \item \textbf{Anes Hodza} (Kandidat)
  \item \textbf{Touseef Arif} (HEX)
\end{itemize}

\subsection{Ablauf des Meetings}
Zu Beginn der Besprechung fragte der HEX den Kandidaten, wie die IPA vorangeht. Daraufhin präsentierte der Kandidat dem HEX
den aktuellen Stand des Codes und dessen Funktionalität. Der HEX war mit dem Fortschritt zufrieden und forderte den
Kandidaten auf, den Source Code zu zeigen. Da ging es primär darum, dass der HEX den Ablauf versteht und die
automatisierten Tests sieht. \newline
Als Nächstes stellte der Kandidat dem HEX den momentanen Stand der Dokumentation vor. Dort wies der HEX den Kandidaten
darauf hin, dass Use-Case Diagramme sich auf die Bewertung positiv auswirken und er diese unbedingt noch erstellen soll.
Ausserdem sagte der HEX, dass es nicht nötig wäre den gesamten Source Code am Ende der Dokumentation zu zeigen, sondern einfach
ihn als ZIP-File beizulegen. \newline
Schlussendlich wurden folgende Daten abgesprochen: Abgabezeitpunkt und Präsentationstermin. Damit wurde sichergestellt, dass
beide Parteien die richtigen Termine kennen.
