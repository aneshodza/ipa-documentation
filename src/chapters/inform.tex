% see A1.1a
\chapter{Informieren}

Die nachfolgende Dokumentation baut auf der Vorlage \cite{Buhler_ipa-template_2022} auf.

\section{Projektmanagement}

Das Projekt wird nach der Projektmanagementmethode IPERKA abgewickelt. Diese Methode passt zum Auftrag, weil der Auftrag in einer Iteration innerhalb von 10 Tagen wasserfallartig realisiert werden soll.

\subsection{Arbeitspakete}

Der Auftrag kann in folgende Arbeitspakete aufgeteilt und nach den Phasen der Projektmanagementmethode gegliedert werden:

\begin{itemize}
    \item Informieren
    \begin{description}
        \item[AP1: Anforderungen analysieren] Der Auftrag wird analysiert und daraus einzelne Arbeitspakete abgeleitet. 
        \item[AP2: Kurzfassung ableiten und schreiben] Es wird anhand von der Analyse eine Kurzfassung des Auftrages abgeleitet und verfasst.
        \item[AP3: Firmenstandards] Die Firmenstandards werden dokumentiert.
    \end{description}
    \item Planen
    \begin{description}
        \item[AP4: Zeitplan gestalten] Es wird anhand der Arbeitspakete ein Zeitplan gestaltet.
        \item[AP5: Gestalten von Diagrammen] Verschiedene Diagramme werden für die bildliche Darstellung des Systemes gestaltet.
        \item[AP6: Testkonzept] Testfälle sowie verschiedene Arten zu testen werden festgehalten und genau dokumentiert.
    \end{description}
    \item Entscheiden
    \begin{description}
        \item[AP7: Bewerten von verschiedenen Optionen] Alle möglichen Optionen werden (falls es eine Entscheidung zum Treffen gibt) ausgewertet. Die Auswertemethode kann je nach Anwendungsfall verschieden sein.
        \item[AP6: Entscheiden der Optionen] Nachdem ausgewertet wurde, wird eine der Optionen genommen. Dies wird dokumentiert.
    \end{description}
    \item Realisieren
    \begin{description}
        \item[AP7: Aufsetzen des Projektes] Das Projekt wird aufgesetzt und mit GitHub verbunden. Das Gleiche gilt für die CI. \newline
        Es werden auch die nötigen \gls{Hooks} auf allen nötigen Diensten eingestellt.
        \item[AP10: Umsetzen der Datenstruktur auf Basis der Diagramme] Die Datenstruktur wird im Projekt umgesetzt. Diese soll nicht von den Diagrammen abweichen.
        \item[AP11: Daten erstellen und manipulieren] Das erstellen, sowie manipulieren von Daten anhand der Webhooks soll umgesetzt werden.
        \item[AP12: Implementieren der \gls{Business Logic}] Die Business Logic wird implementiert. Damit ist das Auswerten der Daten gemeint.
        \item[AP13: Erstellen der \gls{Views}] Die Views werden anhand der \gls{Mockups} erstellt.
    \end{description}
    \item Kontrollieren
    \begin{description}
        \item[AP14: Nach Testkonzept testen] Das Testkonzept wird als Checkliste verwendet und die einzelnen Tests werden nach Protokoll durchgeführt \newline
        Das individuelle Kriterium 7 (Automatisierte Tests) wird auch dokumentiert.
        \item[AP15: Kopplung vom Hauptprogramm] In retrospektive wird festgehalten wie gut das Programm vom Hauptprogramm abgekoppelt ist.
    \end{description}
    \item Auswerten
    \begin{description}
        \item[AP16: Auswerten] Es wird eine Auswertung/Reflexion über das Projekt geschrieben.
    \end{description}
\end{itemize}

\section{Systemaufbau}
Im Folgenden wird die Einbettung des Systems in das Gesamtsystem gezeigt, sowie die vorhandenen Schnittstellen und Akteure beschrieben.

\subsection{Gesamtsystem}

\subsection{Schnittstellen}

\subsection{Akteure}
