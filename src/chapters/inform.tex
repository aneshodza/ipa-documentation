% see A1.1a
\chapter{Informieren}

Die nachfolgende Dokumentation baut auf der Vorlage \cite{Buhler_ipa-template_2022} auf.

\section{Projektmanagement}

Das Projekt nach der Projektmanagementmethode IPERKA abgewickelt. Diese Methode passt zum Auftrag, weil der Auftrag in einer Iteration innerhalb von 10 Tagen wasserfallartig realisiert werden soll.

\subsection{Arbeitspakete}

Der Auftrag kann ich folgende Arbeitspakete aufgeteilt und nach den Phasen der Projektmanagementmethode gegliedert werden:

\begin{itemize}
    \item Informieren
    \begin{description}
        \item[AP1: Anforderungen analysieren] Der Auftrag wird analysiert und daraus einzelne Arbeitspakete abgeleitet. 
        \item[AP2: Lorem] \lipsum[2][1]
    \end{description}
    \item Planen
    \begin{description}
        \item[AP3: Lorem] \lipsum[2][2] 
        \item[AP4: Lorem] \lipsum[2][3]
    \end{description}
    \item Entscheiden
    \begin{description}
        \item[AP5: Lorem] \lipsum[2][4] 
        \item[AP6: Lorem] \lipsum[2][5]
    \end{description}
    \item Realisieren
    \begin{description}
        \item[AP7: Lorem] \lipsum[2][6] 
        \item[AP8: Lorem] \lipsum[2][7]
    \end{description}
    \item Kontrollieren
    \begin{description}
        \item[AP9: Lorem] \lipsum[2][8] 
        \item[AP10: Lorem] \lipsum[2][9]
    \end{description}
    \item Auswerten
    \begin{description}
        \item[AP11: Lorem] \lipsum[2][10] 
        \item[AP12: Lorem] \lipsum[2][11]
    \end{description}
\end{itemize}

\section{Systemaufbau}
Im Folgenden wird die Einbettung des Systems in das Gesamtsystem gezeigt, sowie die vorhandenen Schnittstellen und Akteure beschrieben.

\subsection{Gesamtsystem}

\subsection{Schnittstellen}

\subsection{Akteure}