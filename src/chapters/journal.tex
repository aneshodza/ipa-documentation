% see A2.1
% see B6.2c
\chapter{Arbeitsjournal}

\section{Tag 1}
\begin{tabularx}{\textwidth}[H]{|c|X|}
  \hline
  Gearbeitete Zeit & 8.92h \\ \hline
  Erledigte Arbeiten & Heute wurde die PA gestartet. Es wurde damit begonnen, die Dokumentation anhand vom Template
  \cite{Buhler_ipa-template_2022} zu erstellen. Danach wurden allgemeine Informationen vom PkOrg kopiert, wie zum
  Beispiel die Deklarationen (Kapitel \ref{chap:declaration}).
  Dann wurde der Zeitplan mit den Arbeitspaketen erstellt, die Zusammenfassung der Aufgabe und die
  Firmenstandards. \\ \hline
  Ungeplante Arbeiten & Keine. \\ \hline
  Erfolge & Die Basis für die Dokumentation steht bereits und wurde in Vereinbarung mit dem Zeitplan erledigt. Ausserdem stellte
  sich heraus, dass LaTeX eine sehr gute Wahl für die Dokumentation ist.
  \\ \hline
  Misserfolge & Das Korrigieren von Rechtschreibfehlern geht länger als vorher geplant.  \\ \hline
  Hilfestellungen & Die LaTeX Dokumentation von Overleaf \cite{overleaf} wurde genutzt, um bei LaTeX besser durchblicken zu
  können \\ \hline
\end{tabularx}

\newpage

\section{Tag 2}
\begin{tabularx}{\textwidth}[H]{|c|X|}
  \hline
  Gearbeitete Zeit & 8.50h \\ \hline
  Erledigte Arbeiten & Heute ging es primär darum die Diagramme unter Kapitel \ref{chap:plan}
  (Planen) zu erstellen. Daneben ging es noch ein bisschen darum, die Arbeit von gestern
  aufzuräumen. Einige Punkte wurden übersehen, wie zum Beispiel die Erklärung von zwei
  externen Schnittstellen. Zu Ende wurde allgemein an der Dokumentation gearbeitet, wie
  im Zeitplan eingetragen. \\ \hline
  Ungeplante Arbeiten & Unerwartet begann die Arbeit am Testkonzept bereits. Das passierte,
  da der VF eine Frage gestellt werden musste, aber diese am Vormittag nicht erreichbar war. Nachdem
  die Unklarheiten aus dem Web geräumt wurden, ging die Arbeit wie geplant weiter.
  \\ \hline
  Erfolge & Dank der Arbeit am Testkonzept entstand ein kleiner Vorsprung zum
  Zeitplan machen. Die gewonnene Zeit kann gut als Puffer beim Implementieren genutzt werden.
  \\ \hline
  Misserfolge & Heute gab es keine Misserfolge \\ \hline
  Hilfestellungen & Die VF erklärte die Software-Design Anforderung 3: \enquote{Die 
  Informationen müssen beim öffnen der Ticket-Ansicht schon verfügbar sein}.  \\ \hline
\end{tabularx}

\newpage

\section{Tag 3}
\begin{tabularx}{\textwidth}[H]{|c|X|}
  \hline
  Gearbeitete Zeit & 8.75h \\ \hline
  Erledigte Arbeiten & Der Tag begann damit, die Diagramme vom Vortag leicht auszubessern. Zu
  Beginn wurde das Activity Diagramm unter \ref{fig:activity_hook_call} angepasst, da es nicht
  ganz realitätsgetreu war. Dann wurde noch ein Vorschlag für das ERD unter
  \ref{fig:erd_has_many_through} hinzugefügt. Die ERD Anpassung ergab sich aus der Anpassung
  vom Activity Diagramm\newline
  Danach wurde das Testkonzept unter Kapitel \ref{sec:testkonzept} finalisiert. Da Arbeit bereits
  am Vortag begonnen wurde, mussten nur die manuellen Testfälle (\ref{sec:manual-tests}) noch erstellt
  werden. \newline
  Dann wurden wieder Änderungen an den Diagrammen vorgenommen. Als erstes wurde ein neues Mockup
  (Grafik \ref{fig:mockup_sublists}) hinzugefügt, welches den dritten (neuen) Vorschlag für das ERD
  besser darstellt. Dank den neuen Informationen ergab sich auch, dass das geplante Activity Diagramm
  für die SemaphoreCI Webhook eventuell nicht umsetzbar ist, weshalb ein \enquote{Fallback Plan}
  erstellt wurde. Dieser wurde mit Grafik \ref{fig:activity_plan_b} dargestellt. \newline
  Dann wurde endlich am Bewerten gearbeitet. Dieses Kapitel wurde auch schon abgeschlossen, weshalb
  das Entscheiden schon zum Teil erledigt werden konnte. \newline
  Zum Schluss war der Expertenbesuch, welcher unter \ref{sec:first-expert-meeting} dokumentiert wurde.
  Da sich beim Meeting herausstellte, dass die Anforderung zum Glossar nicht ganz verstanden wurden,
  wurde diese nochmals angepasst. \\ \hline
  Ungeplante Arbeiten & Es wurde viel Arbeit an den Diagrammen erledigt, was so nicht geplant war. Ausserdem
  begann die Arbeit am Entscheiden selbst, da das Auswerten erledigt wurde. Beides war für den nächsten Tag
  geplant. \\ \hline
  Erfolge & Obwohl viel vom Vortag korrigiert wurde, war es möglich einen Vorsprung zum Zeitplan aufzubauen.
  Ein anderer Erfolg ist, dass das der erste Expertenbesuch erledigt wurde und der HEX zufrieden scheint. \\ \hline
  Misserfolge & Viele Diagramme mussten geändert werden, da neue Informationen erlangt wurden. Das ein backup
  Plan für die SemaphoreCI Webhooks erstellt werden musste ist auch ein bisschen Sorgen bereitend. \\ \hline
  Hilfestellungen & Der HEX wurde nach einer Erläuterung zum Glossar gefragt. \\ \hline
\end{tabularx}

\newpage

\section{Tag 4}
\begin{tabularx}{\textwidth}[H]{|c|X|}
  \hline
  Gearbeitete Zeit & 8.58h \\ \hline
  Erledigte Arbeiten & Zu Beginn des Tages wurden das Kapitel \enquote{Entscheiden} beendet. Das ging länger als geplant,
  weshalb der Fortschritt wieder mit dem Zeitplan übereinstimmt. \newline
  Gegen den Mittag wurde dann Plan B wegen SemaphoreCI mit der VF besprochen und angeschaut. Die VF war einverstanden mit 
  dem Plan, doch wies darauf hin, dass die Abfrage von GitHub gut begründet werden muss. \newline
  Am Nachmittag begann die Implementation. Das Projekt wurde erfolgreich erstellt und aufgesetzt. Das wurde leicht vor dem
  Zeitplan erledigt, weshalb die Implementation der ERDs begonnen wurde. \\ \hline
  Ungeplante Arbeiten & Die Implementation der ERDs begann bereits, obwohl das erst für morgen geplant wurde. \\ \hline
  Erfolge & Die VF segnete Plan B ab, was eine grosse Erleichterung war. Am Nachmittag begann endlich die Implementation,
  was als grosser Erfolg gesehen werden kann. \\ \hline
  Misserfolge & Das Entscheiden ging über vier Stunden, anstatt die geplanten zwei, was zum Verlust von wertvollem
  Vorsprung zum Zeitplan führte. \\ \hline
  Hilfestellungen & Die VF wurde kontaktiert, da es Unsicherheiten bei Plan B gab. Ein altes Repository, welches
  ein funktionierendes database.yml enthält, wurde auch verwendet. \\ \hline
\end{tabularx}

\newpage

\section{Tag 5}
\begin{tabularx}{\textwidth}[H]{|c|X|}
  \hline
  Gearbeitete Zeit & 7h \\ \hline
  Erledigte Arbeiten & Heute wurde zu Beginn am Implementieren der Datenstruktur gearbeitet. Diese Arbeit ging jedoch
  viel länger als geplant, da vieles vom Setup nicht richtig funktionierte. Als Hilfe wurde auch die VF dazu geholt,
  welche Unklarheiten zur Aufgabenstellung sowie zur Korrektheit der Implementation. \newline
  Am Nachmittag wurde dann bereits am Erstellen und Manipulieren der Daten gearbeitet, da die geplante Zeit für das 
  Schreiben der Dokumentation nicht vollständig nötig war. \\ \hline
  Ungeplante Arbeiten & Die Arbeit am Erstellen und Manipulieren der Daten begann bereits. \\ \hline
  Erfolge & Dank dem Puffer ist die Implementation bereits ein bisschen weiter als geplant \\ \hline
  Misserfolge & Das Implementieren des ERD nahm sehr viel mehr Zeit in Anspruch, da das Setup unvollständig war \\ \hline
  Hilfestellungen & Die VF wurde kurz dazu geholt, um Unklarheiten zu klären \\ \hline
\end{tabularx}

\newpage

\section{Tag 6}
\begin{tabularx}{\textwidth}[H]{|c|X|}
  \hline
  Gearbeitete Zeit & 7.75h \\ \hline
  Erledigte Arbeiten & Der Tag begann damit, das bereits am Freitag begonnene AP \enquote{Daten erstellen und
  manipulieren}. Da aber die Deploy Objekte so aufwändig waren, mündete diese Arbeit in das nächste AP
  \enquote{Implementieren der Business Logik}, weshalb die Pull Request auch mit \enquote{Create \enquote{creation
  and manipulation of data} and basic business logic} betitelt wurde. Am Nachmittag wurde dann auch die Business-Logik
  fertig implementiert sowie dokumentiert. Da beide AP sehr eng miteinander hängen wurden beide unter dem gleichen
  Kapitel dokumentiert. \\ \hline
  Ungeplante Arbeiten & Die Business Logik hätte heute erst begonnen werden sollen, anstatt dieses AP bereits zu
  beenden. Damit kann ein grosser Teil von morgen für das Aufbessern der Codebase genutzt werden. \\ \hline
  Erfolge & Das Implementieren der Deploy Objekte und dessen Erstellung ging um einiges einfacher als geplant. \\ \hline
  Misserfolge & Plan A für die SemaphoreCI Webhooks war leider nicht möglich, weshalb gepollt werden muss bei externen
  Diensten. Das wird unter \ref{sec:why_active_polling} genauer beschrieben. \\ \hline
  Hilfestellungen & Keine. \\
  \hline
\end{tabularx}

\newpage

\section{Tag 7}
\begin{tabularx}{\textwidth}[H]{|c|X|}
  \hline
  Gearbeitete Zeit & 8.75h \\ \hline
  Erledigte Arbeiten & Heute ging es darum, die Views des Plugins zu erstellen, was ein bisschen länger als den geplanten
  Zeitraum in Anspruch nahm. Es wurden zwischendurch Ausbesserungen an der Dokumentation, sowie verbessern des Test-Setups
  erledigt. \\ \hline
  Ungeplante Arbeiten & Das Test-Setup erneut noch einmal überarbeiten zu müssen war nicht geplant. Ausserdem wurde die
  Dokumentation (dieses Dokument, sowie das Readme des Projektes) an bestimmten Orten ausgebessert, was auch nicht so im
  Zeitplan vorgesehen war. \\ \hline
  Erfolge & Glücklicherweise war die Implementation der Views nicht so schwierig wie gedacht. Das Problem war mehr, diese
  schön zu formatieren, da alles in Strings gemacht werden musste. \\ \hline
  Misserfolge & Das das Test-Setup erneut Probleme machte war schade. Es nahm sehr viel Zeit in Anspruch das wieder hin zu
  kriegen. \\ \hline
  Hilfestellungen & Die VF wurde beim Reparieren des Test-Setups kurz dazu geholt, da das Problem recht komplex erschien.
  Es stellte sich aber raus, dass nur das Folgen einer Konvention für die Methoden-Namen der Fehler war. \newline
  Ausserdem wurde die VF allgemein zur IPA ein bisschen ausgefragt (zum Kriterium \enquote{Selbstständiges Arbeiten}
  primär) \\
  \hline
\end{tabularx}

\newpage

\section{Tag 8}
\begin{tabularx}{\textwidth}[H]{|c|X|}
  \hline
  Erledigte Arbeiten & \lipsum[23] \\ \hline
  Ungeplante Arbeiten & \lipsum[24] \\ \hline
  Erfolge & \lipsum[25] \\ \hline
  Misserfolge & \lipsum[26] \\ \hline
  Hilfestellungen & \lipsum[27] \\
  \hline
\end{tabularx}

\newpage

\section{Tag 9}
\begin{tabularx}{\textwidth}[H]{|c|X|}
  \hline
  Erledigte Arbeiten & \lipsum[23] \\ \hline
  Ungeplante Arbeiten & \lipsum[24] \\ \hline
  Erfolge & \lipsum[25] \\ \hline
  Misserfolge & \lipsum[26] \\ \hline
  Hilfestellungen & \lipsum[27] \\
  \hline
\end{tabularx}

\newpage

\section{Tag 10}
\begin{tabularx}{\textwidth}[H]{|c|X|}
  \hline
  Erledigte Arbeiten & \lipsum[23] \\ \hline
  Ungeplante Arbeiten & \lipsum[24] \\ \hline
  Erfolge & \lipsum[25] \\ \hline
  Misserfolge & \lipsum[26] \\ \hline
  Hilfestellungen & \lipsum[27] \\
  \hline
\end{tabularx}
