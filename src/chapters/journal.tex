% see A2.1
% see B6.2c
\chapter{Arbeitsjournal}

\section{Tag 1}
\begin{tabularx}{\textwidth}[H]{|c|X|}
  \hline
  Gearbeitete Zeit & 8.92h \\ \hline
  Erledigte Arbeiten & Heute wurde die PA gestartet. Ich fing damit an, die Dokumentation anhand vom Template
  \cite{Buhler_ipa-template_2022} zu erstellen. Danach kümmerte ich mich um die allgemeinen Informationen und die
  Aufgabenstellung vom PkOrg. Von dort kopierte ich auch verschiedene Texte, wie zum Beispiel die
  Deklarationen (Kapitel \ref{chap:declaration}). \newline
  Dann wurde der Zeitplan mit den Arbeitspaketen erstellt, die Zusammenfassung der Aufgabe und die
  Firmenstandards.
  \\ \hline
  Ungeplante Arbeiten & Keine. \\ \hline
  Erfolge & Die Basis für die Dokumentation steht bereits und ich bin momentan perfekt auf dem Zeitplan. \newline
  Ausserdem komme ich massiv besser mit LaTeX zurecht, als ich erwartet habe.
  \\ \hline
  Misserfolge & Das Korrigieren von Rechtschreibfehlern geht länger als vorher geplant.  \\ \hline
  Hilfestellungen & Die LaTeX Dokumentation von Overleaf \cite{overleaf} war mir eine grosse
  Hilfe, da ich noch eher unbekannt bin mit LaTeX. \\
  \hline
\end{tabularx}

\newpage

\section{Tag 2}
\begin{tabularx}{\textwidth}[H]{|c|X|}
  \hline
  Gearbeitete Zeit & 8.50h \\ \hline
  Erledigte Arbeiten & Heute ging es primär darum die Diagramme unter Kapitel \ref{chap:plan}
  (Planen) zu erstellen. Daneben ging es noch ein bisschen darum, die Arbeit von gestern
  aufzuräumen. Einige Punkte wurden übersehen, wie zum Beispiel die Erklärung von zwei
  externen Schnittstellen. Zu Ende wurde allgemein an der Dokumentation gearbeitet, wie
  im Zeitplan eingetragen. \\ \hline
  Ungeplante Arbeiten & Unerwartet begann die Arbeit am Testkonzept bereits. Das passierte,
  da ich eine Frage an die VF hatte, aber diese am Vormittag nicht erreichbar war. Nachdem
  die Unklarheiten aus dem Web geräumt wurden, ging die Arbeit wie geplant weiter.
  \\ \hline
  Erfolge & Dank der Arbeit am Testkonzept konnte ich mir einen kleinen Vorsprung zum
  Zeitplan machen. Die gewonnene Zeit kann gut als Puffer beim Implementieren genutzt werden.
  \\ \hline
  Misserfolge & Heute gab es keine Misserfolge \\ \hline
  Hilfestellungen & Die VF erklärte mir Software-Design Anforderung 3: \enquote{Die 
  Informationen müssen beim öffnen der Ticket-Ansicht schon verfügbar sein}.  \\ \hline
\end{tabularx}

\newpage

\section{Tag 3}
\begin{tabularx}{\textwidth}[H]{|c|X|}
  \hline
  Erledigte Arbeiten & \lipsum[23] \\ \hline
  Ungeplante Arbeiten & \lipsum[24] \\ \hline
  Erfolge & \lipsum[25] \\ \hline
  Misserfolge & \lipsum[26] \\ \hline
  Hilfestellungen & \lipsum[27] \\
  \hline
\end{tabularx}

\newpage

\section{Tag 4}
\begin{tabularx}{\textwidth}[H]{|c|X|}
  \hline
  Erledigte Arbeiten & \lipsum[23] \\ \hline
  Ungeplante Arbeiten & \lipsum[24] \\ \hline
  Erfolge & \lipsum[25] \\ \hline
  Misserfolge & \lipsum[26] \\ \hline
  Hilfestellungen & \lipsum[27] \\
  \hline
\end{tabularx}

\newpage

\section{Tag 5}
\begin{tabularx}{\textwidth}[H]{|c|X|}
  \hline
  Erledigte Arbeiten & \lipsum[23] \\ \hline
  Ungeplante Arbeiten & \lipsum[24] \\ \hline
  Erfolge & \lipsum[25] \\ \hline
  Misserfolge & \lipsum[26] \\ \hline
  Hilfestellungen & \lipsum[27] \\
  \hline
\end{tabularx}

\newpage

\section{Tag 6}
\begin{tabularx}{\textwidth}[H]{|c|X|}
  \hline
  Erledigte Arbeiten & \lipsum[23] \\ \hline
  Ungeplante Arbeiten & \lipsum[24] \\ \hline
  Erfolge & \lipsum[25] \\ \hline
  Misserfolge & \lipsum[26] \\ \hline
  Hilfestellungen & \lipsum[27] \\
  \hline
\end{tabularx}

\newpage

\section{Tag 7}
\begin{tabularx}{\textwidth}[H]{|c|X|}
  \hline
  Erledigte Arbeiten & \lipsum[23] \\ \hline
  Ungeplante Arbeiten & \lipsum[24] \\ \hline
  Erfolge & \lipsum[25] \\ \hline
  Misserfolge & \lipsum[26] \\ \hline
  Hilfestellungen & \lipsum[27] \\
  \hline
\end{tabularx}

\newpage

\section{Tag 8}
\begin{tabularx}{\textwidth}[H]{|c|X|}
  \hline
  Erledigte Arbeiten & \lipsum[23] \\ \hline
  Ungeplante Arbeiten & \lipsum[24] \\ \hline
  Erfolge & \lipsum[25] \\ \hline
  Misserfolge & \lipsum[26] \\ \hline
  Hilfestellungen & \lipsum[27] \\
  \hline
\end{tabularx}

\newpage

\section{Tag 9}
\begin{tabularx}{\textwidth}[H]{|c|X|}
  \hline
  Erledigte Arbeiten & \lipsum[23] \\ \hline
  Ungeplante Arbeiten & \lipsum[24] \\ \hline
  Erfolge & \lipsum[25] \\ \hline
  Misserfolge & \lipsum[26] \\ \hline
  Hilfestellungen & \lipsum[27] \\
  \hline
\end{tabularx}

\newpage

\section{Tag 10}
\begin{tabularx}{\textwidth}[H]{|c|X|}
  \hline
  Erledigte Arbeiten & \lipsum[23] \\ \hline
  Ungeplante Arbeiten & \lipsum[24] \\ \hline
  Erfolge & \lipsum[25] \\ \hline
  Misserfolge & \lipsum[26] \\ \hline
  Hilfestellungen & \lipsum[27] \\
  \hline
\end{tabularx}
