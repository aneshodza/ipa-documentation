% See B1
\chapter{Kurzfassung}

Die Kurzfassung gibt einen Überblick über das vorliegende Projekt.

\section{Ausgangssituation}

Die Renuo AG verwendet schon seit längerem Redmine, ein \gls{Open-Source} Projektmanagment tool. Es dient dazu, einem
Team mehrere Projekte zu verwalten und den überblick zu behalten. Es hat sehr viele features, wie zum Beispiel: \cite{redmine_homepage}
\begin{itemize}
    \item Unterztützung von mehreren Projekten
    \item Flexibles Rollensystem
    \item \textbf{Issue-Tracking}
    \item ...
\end{itemize}
Für uns wichtig ist das \enquote{Issue-Tracking}. Dieses ermöglicht es einem Nutzer jedem Projekt verschiedene Issues zuzuweisen.
Diese Issues sollen aber nicht als Bugs verstanden werden, sondern als Aufgaben, die erledigt werden müssen. In der Regel sind das
\gls{Feature-Requests}, die von Kunden kommen. Diese werden dann von einem Entwickler übernommen und bearbeitet. \newline
Das Problem ist, dass es bei aktiven Projekten sehr viele Issues gibt und es schwierig ist, den Überblick zu behalten.
Das Ziel dieser PA ist es, die \gls{Pull-Request}s sowie die \gls{Deployment}-Status von Issues direkt im Redmine anzuzeigen.

\section{Umsetzung}

Die PA wird mit IPERKA geplant und umgesetzt. \newline
Softwaretechnisch wird die PA mit Ruby on Rails umgesetzt, da Redmine auf dieser Technologie basiert. Es werden auch
die gleichen Frameworks für: das Testen, die Views und so weiter wie im standard Redmine verwendet. \newline
Das Plugin soll auf Webhooks der externen Dienste reagieren und bestimmte Logik als Reaktion darauf ausführen. \newline
Während der Entwicklung werden für das Plugin auch automatisierte Tests geschrieben, um die Qualität des Codes zu gewährleisten.
Die Test-Abdeckung soll 100\% betragen.

\section{Ergebnis}

Das Ergebnis dieser PA sollte ein Redmine-Plugin sein, welches die open gennanten Informationen wiedergeben kann. \newline
Das Plugin soll komplett von Redmine abgekoppelt sein, damit es auch in anderen Redmine-Instanzen verwendet werden kann.
Das bedeutet, dass der Quellcode von Redmine selbst nicht verändert werden darf. \newline
