% See B1
\chapter{Kurzfassung}

Die Kurzfassung gibt einen Überblick über das vorliegende Projekt.

\section{Ausgangssituation}

Die Renuo AG verwendet schon seit längerem Redmine, ein Open-Source Projektmanagement tool. Es dient dazu, einem
Team mehrere Projekte zu verwalten und den Überblick zu behalten. Es hat sehr viele features, wie zum Beispiel: \cite{redmine_homepage}
\begin{itemize}
    \item Unterstützung von mehreren Projekten
    \item Flexibles Rollensystem
    \item \textbf{Issue-Tracking}
    \item ...
\end{itemize}
Für uns wichtig ist das Issue-Tracking. Dieses ermöglicht es einem Nutzer jedem Projekt verschiedene Issues zuzuweisen.
Diese Issues sollen aber nicht als Bugs verstanden werden, sondern als Aufgaben, die erledigt werden müssen. In der Regel sind das
Feature-Requests, die von Kunden kommen. Diese werden dann von einem Entwickler übernommen und bearbeitet. \newline
Das Problem ist, dass es bei aktiven Projekten sehr viele Issues gibt und es schwierig ist, den Überblick zu behalten.
Das Ziel dieser PA ist es, die Pull Requests sowie die Deployment-Status von Issues direkt im Redmine anzuzeigen.

\section{Umsetzung}

Die PA wurde mit IPERKA geplant und umgesetzt. \newline
Softwaretechnisch ist die PA mit Ruby on Rails umgesetzt worden, da Redmine auf dieser Technologie basiert. Es wurden auch
die gleichen Frameworks für: das Testen, die Views und so weiter wie im Standard Redmine verwendet. \newline
Das Plugin reagiert auf Webhooks der externen Dienste und führt bestimmte Logik als Reaktion darauf aus. \newline
Während der Entwicklung wurden für das Plugin auch automatisierte Tests geschrieben, um die Qualität des Codes zu gewährleisten.
Die Test-Abdeckung betrug bei jeder Pull Request 100\%.

\section{Ergebnis}

Das Ergebnis dieser PA ist ein Deployment-Ready Plugin, welches Pull Requests und dessen Deployments in Unterlisten eines
Issues auflistet. \newline
Das Plugin ist vollständig von Redmine abgekoppelt und kann somit in jeder Redmine-Instanz verwendet werden. \newline

