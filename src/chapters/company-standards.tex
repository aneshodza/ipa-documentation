\chapter{Firmenstandards}
Die Renuo AG verwendet verschiedene Tools zur Garantie der Codequalität. Diese Tools werden in diesem Kapitel dokumentiert.

\section{Code-Analyse}
Für sauberes Formatieren werden je nach Sprache unterschiedliche \gls{Linter} verwendet. Nebenbei können auch andere Tools, welche die Sicherheit verbessern genutzt werden. In diesem Falle werden folgende Tools verwendet:

\begin{itemize}
    \item \textbf{Rubocop} wird für Ruby Quellcode verwendet.
    \item \textbf{Brakeman} wird für das Vermeiden von groben Sicherheitslücken verwendet.
\end{itemize}

\section{Tests}
Um die Funktionalität des Programmes zu garantieren, werden automatisierte Tests geschrieben. Dabei soll jede einzelne Zeile (falls die Klasse über fünf Zeilen hat) getestet werden.

\section{Code Review}
Um die Qualität des Codes zu garantieren, wird dieser von einem anderen Entwickler überprüft. Da dieses Projekt eine Einzelarbeit ist, wird der Quellcode vom Kandidaten selber überprüft.
