\chapter{Arbeitsumfeld}

\section{Firmenstandards}
Die Renuo AG verwendet verschiedene Tools zur Garantie der Codequalität. Diese Tools werden in diesem Kapitel dokumentiert.
\subsection{Code-Analyse}
Für sauberes Formatieren werden je nach Sprache unterschiedliche \gls{Linter} verwendet. Nebenbei können auch andere Tools, welche die Sicherheit verbessern, genutzt werden. In diesem Falle werden folgende Tools verwendet:
\begin{itemize}
    \item \textbf{Rubocop} wird für Ruby Quellcode verwendet.
    \item \textbf{Brakeman} wird für das Vermeiden von groben Sicherheitslücken verwendet.
\end{itemize}
\subsection{Tests}
Um die Funktionalität des Programmes zu garantieren, werden automatisierte Tests geschrieben. Dabei soll jede einzelne Zeile (falls die Klasse über fünf Zeilen hat) getestet werden.
\subsection{Code Review}
Um die Qualität des Codes zu garantieren, wird dieser von einem anderen Entwickler überprüft. Da dieses Projekt eine Einzelarbeit ist, wird der Quellcode vom Kandidaten selber überprüft.

\section{Entwicklungsumgebung}
\subsection{Versionierung}
Für die Versionierung wird Git verwendet. Dabei wird GitHub als Remote-Repository verwendet. Das Repository mit
dem Source-Code kann unter \url{https://github.com/aneshodza/gnosis} gefunden werden. Für eine genauere Beschreibung
der Versionierung siehe Kapitel \ref{sec:versioning}.
\subsection{IDE}
Als IDE wird vim mit verschiedenen Plugins verwendet. Bestimmte Sachen wurden in der \bgmintinline{bash}{.vimrc} Datei
konfiguriert, damit die Arbeit möglichst effizient ist. \newline
Diese Konfigurationen sind unter \newline
\url{https://github.com/aneshodza/.dotfiles/blob/ad87ee9ecc5588a59d66e211797792099569ca95/.vimrc} zu finden.
\subsection{CI/CD}
Für die CI/CD Pipeline wird SemaphoreCI verwendet. Das ist passend, da auch die PA sehr eng mit SemaphoreCI verbunden
ist.
