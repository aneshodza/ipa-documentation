% [2]: Seite 11
\chapter{Deklaration}

Folgender Abschnitt beschreibt die Vorkenntnisse des Kandidaten und dessen Vorbereitung.

\section{Vorkenntnisse}

Der Lernende hat seit Praktikumsbeginn (1. August 2022) mit Ruby on Rails gearbeitet. Auch wurden die Tests in dieser Zeit mit RSpec und Capybara geschrieben und die Code-Qualität mit Rubocop überprüft. Ebenfalls seit Praktikumsstart wurde Git und Git Flow zusammen mit Github eingesetzt. Code-Reviews gehören ebenfalls zur täglichen Arbeit (sowohl Code Reviews durchführen wie auch entgegennehmen). Heroku und SemaphoreCI werden ebenfalls seit Praktikumsbeginn eingesetzt.

\section{Vorarbeiten}

Folgende Arbeiten bereiten die IPA vor: \newline
- Vanilla-Redmine wird lokal eingerichtet \newline
- Boilerplate Redmine-Plugin wird lokal so aufgesetzt, dass End-to-end-Tests mit Capybara gegen Redmine ausgeführt werden können. \newline
- Man liest sich in die Redmine-Plugin-Dokumentation ein: \url{https://www.redmine.org/boards/4/topics/45309} \newline
- Man schreibt einen MiniTest-Case für das Opensource-Projekt \url{https://gfrör.li/} (\url{https://github.com/gfroerli/api}) \newline
- Man liest über Webhooks: \url{https://docs.semaphoreci.com/essentials/webhook-notifications/}, \url{https://docs.github.com/developers/webhooks-and-events/webhooks/webhook-events-and-payloads\#pull\_request} \newline

\subsection{Andere Vorarbeiten}
Ausserhalb der von PkOrg festgehaltenen Vorarbeiten, wurden folgende Vorarbeiten geleistet:
\begin{itemize}
    \item Das IPA-Latex template von R. Bühler auf GitHub wurde verwendet \cite{Buhler_ipa-template_2022}
\end{itemize}

\section{Neue Lerninhalte}

Der Lernende hat schon sehr viele automatisierte Tests mit RSpec geschrieben. Bei diesem Projekt wird jedoch das Standard-Test-Framework von Rails namens MiniTest eingesetzt. Im Gegensatz zum DSL-Ansatz von RSpec orientiert sich MiniTest eher am klassenbasierten Ansatz von JUnit. MiniTest ist also einfacher, aber weniger mächtig.

\section{Arbeiten in den letzten 6 Monaten}

In der letzten Zeit wurde hauptsächlich mit Ruby on Rails gearbeitet (ca. 150 Tage, Stand 12. April). Das grösste Projekt war ein Dashboard für eine Firma, die Solaranlagen verkauft in Regie des Lernenden (unter Anleitung). Dazu wurde Rails verwendet.
