% see A1.4
\chapter{Planen}

Die Zeitplanung wird in der Abbildung \ref{fig:timeplan} oberhalb gezeigt. Die restlichen Aspekte der Planung sind in diesem Kapitel dokumentiert.

\begin{figure}[H]
  \begin{center}
    \begin{tikzpicture}
      \begin{umlsystem}[x=0, y=0]{PkOrg}
      \end{umlsystem}
      \umlactor[x=-5, y=.1]{Verantwortliche Fachperson}
      \umlactor[x=5, y=.1]{Hauptexperte}
      \umlassoc{Verantwortliche Fachperson}{PkOrg}
      \umlassoc{Hauptexperte}{PkOrg}
    \end{tikzpicture}
  \end{center}
  \caption[\enquote{Systemkontextdiagramm} erstellt mit Tikz UML]{Systemkontextdiagramm}
  \label{fig:systemcontext}
\end{figure}

\section{Diagramme}
In diesem Abschnitt werden verschiedene Diagramme, welche zur bildlichen Darstellung des Systems dienen sollen.
\subsection{Entity-Relationship-Diagram}
\subsection{Activity-Diagram}

\section{Testkonzept}
\label{sec:testkonzept}
Das Testkonzept beschreibt, wie und mit welchen Werkzeugen das Resultat auf seine Richtigkeit kontrolliert wird.

\subsection{Testmethoden}
\subsubsection{Manuelle Tests}
\subsubsection{Automatisierte Tests}

\subsection{Testmittel}
