% see A1.4
\chapter{Planen}

Die Zeitplanung wird in der Abbildung \ref{fig:timeplan} oberhalb gezeigt. Die restlichen Aspekte der Planung sind in diesem Kapitel dokumentiert.

\begin{figure}[H]
  \begin{center}
    \begin{tikzpicture}
      \begin{umlsystem}[x=0, y=0]{PkOrg}
      \end{umlsystem}
      \umlactor[x=-5, y=.1]{Verantwortliche Fachperson}
      \umlactor[x=5, y=.1]{Hauptexperte}
      \umlassoc{Verantwortliche Fachperson}{PkOrg}
      \umlassoc{Hauptexperte}{PkOrg}
    \end{tikzpicture}
  \end{center}
  \caption[\enquote{Systemkontextdiagramm} erstellt mit Tikz UML]{Systemkontextdiagramm}
  \label{fig:systemcontext}
\end{figure}

\section{Diagramme}
In diesem Abschnitt werden verschiedene Diagramme, welche zur bildlichen Darstellung des Systemes dienen sollen.
\subsection{Entity-Relationship-Diagram}
\subsection{Activity-Diagram}

\section{Testkonzept}
Das Testkonzept beschreibt, wie und mit welchen Werkzeugen das Resultat auf seine Richtigkeit kontrolliert wird.

\subsection{Automatisierte Tests}
Es werden automatisierte Tests für das Plugin geschrieben, welche die Funktionalität der Applikation testen.
Diese werden mit den gleichen Frameworks wie die vom Redmine geschrieben. Das heisst, dass die Tests mit
folgenden Frameworks geschrieben werden:
\begin{itemize}
  \item \textbf{MiniTest} für die Unit-Tests
  \item \textbf{Capybara} für die System-Tests
\end{itemize}
Die coverage sollte 100\% betragen (bei Klassen über 5 Zeilen). Diese Tests werden dann automatisch auf 
SemaphoreCI ausgeführt. Von der CI erhalten wir dann einen Coverage-Report, sowie eine Liste der
fehlgeschlagenen Tests. Nur falls alles in Ordnung ist, kann man die Pull-Request mergen.
\subsubsection{Weitere Tools}
Tools, welche nicht von Redmine selbst verwendet werden, welche aber dennoch im Plugin verwendet werden sind:
\begin{itemize}
  \item \textbf{Faker} für das Erstellen von realistischen Testdaten.
  \item \textbf{FactoryBot} für das Initialisieren von Objekten.
\end{itemize}
Die Coverage, sowie andere wichtige Informationen werden unter \ref{sec:automated-tests} dokumentiert.

\subsection{Manuelle Tests}
Während der Entwicklung und für allfällige Demonstrationszwecke werden manuelle Tests durchgeführt. Diese
werden in diesem Kapitel beschrieben und unter \ref{sec:manual-tests} protokolliert.

\subsection{Testmittel}
