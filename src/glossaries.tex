% see B6.6
\newglossaryentry{Organigramm}{
  name={Organigramm},
  description={Ein Organigramm stellt eine Organisation und deren Aufbauorganisation grafisch dar.}
}

\newglossaryentry{Hooks}{
  name={Hooks},
  description={Hooks werden im Programmieren genutzt, um unter bestimmten Bedingungen Funktionen auszuführen.}
}

\newglossaryentry{Business Logic}{
  name={Business Logic},
  description={Die Business Logic ist Software, welche die Logik einer Anwendung implementiert.}
}

\newglossaryentry{Views}{
  name={Views},
  description={Views sind in der Webentwicklung die Darstellungsschicht einer Anwendung.}
}

\newglossaryentry{Mockups}{
  name={Mockups},
  description={Mockups sind in der Softwareentwicklung Entwürfe von Benutzeroberflächen.}
}

\newglossaryentry{Linter}{
  name={Linter},
  description={Ein Linter ist ein Programm, welches Quellcode auf Fehler, wahrscheinliche Bugs und Stilfehler überprüft.}
}

\newglossaryentry{Open-Source}{
  name={Open-Source},
  description={Open-Source Software ist Software, deren Quellcode öffentlich zugänglich und meistens auch bearbeitbar ist.}
}

\newglossaryentry{Feature-Requests}{
  name={Feature-Requests},
  description={Ein Feature-Request ist eine Anfrage an die Entwickler eines Produktes, um ein neues Feature (Funktionalität) zu implementieren.}
}

\newglossaryentry{Pull-Request}{
  name={Pull-Request},
  description={Ein Pull-Request ist eine Anfrage an die Entwickler eines Produktes, um einen neuen Code in das Projekt zu integrieren.}
}

\newglossaryentry{Deployment}{
  name={Deployment},
  description={Deployment ist der Prozess, bei dem Software auf einem Server installiert wird.}
}
